\documentclass[a4paper,10pt]{article}


% Title Page
\title{Network Programming\\
\small{Assignment 4: Web Programming}}
\author{%
        \mbox{}\\
        Thomas Veerman\\
        \texttt{tveerman@cs.vu.nl}\\
        1329545\\
        \mbox{}\\
        Martijn Vermaat\\
        \texttt{mvermaat@cs.vu.nl}\\
        1362917
}


\begin{document}
\maketitle

\section{Conference Web site}
\paragraph{Question 1}
We implemented our \textsc{CGI} files in C, the main reason being that
communication with the paper storage server has to be done through \textsc{Sun RPC}
and C is the language with the main \textsc{Sun RPC} support.

Another language frequently used for \textsc{CGI} programming is Perl,
probably because it is very good at manipulating strings and does not have to
be compiled. We briefly considered using Perl with an \texttt{rpcgen}
implementation for Perl or calls directly to C functions (for
example by using the \texttt{Inline::C} package). Our experience with Perl
however is limited and we decided the possible advantage would not be worth the
efforts of getting any of the above methods to work.

\paragraph{Question 2}
The paper storage server has to be accessed through \textsc{Sun RPC}, so this has
to be possible with the language we use. We are not aware of a \texttt{rpcgen}
implementation generating PHP code and talking directly with a \textsc{Sun RPC}
server using PHP socket seems a pointless and time-consuming excercise. The
other option would be to talk to the paper storage server using a different language
and embedding these calls in the PHP code somehow (e.g. by writing a PHP
extension, or executing a compiled C program. This would require some work.

On the other hand, \textsc{CGI} files can be implemented using any language, so in
that case it is easy to pick a language from which we can communicate with the
paper storage server. However, this does not mean \textsc{CGI} is the only way to
deal with this. There are Apache modules for many languages that allow to
execute a piece of code in such a language whenever a request comes in.

\paragraph{Question 3}
We implemented the upload page using a form and a \textsc{POST} request. In general,
\textsc{GET} requests should be used for pages which are idem potent, where 
\textsc{POST} should be used for page which are not. For example, pages showing
information such as \texttt{papers.cgi} and \texttt{paperview.cgi} are idem potent.

\section{Hotel reservations}
\paragraph{Question 4}
It would have been possible to implement the hotel reservation pages in any language
that supports socket programming.

\section{Program documentation}
\subsection{Conference Web site}
For the Conference Website we have written 3 CGI pages:
\begin{center}
% use packages: array
\begin{tabular}{ l | p{9cm} }
papers.c & a A page that shows the available papers\\ \hline
paperview.c & A page that makes it possible to download papers\\ \hline
paperload.c & A page that provides a form to upload papers\\
\end{tabular}
\end{center}

\paragraph{papers.c}
This program connects to the (local) paperstorage server and executes a \texttt{list}
command in order to retrieve the list of stored papers. It presents them the same way
as our paperclient of the previous assignment, with the addition of making the titles
of the papers hyperlinks to paperview.cgi so one can download the papers.

\paragraph{paperview.c}
This program connects to the (local) paperstorage server and executes a \texttt{fetch}
command in order to retrieve a specific paper. This file is sent to the browser with
the proper mime type, so the browser can decide how to view the file (or provide a
means to save it somewhere). Which paper it retrieves depends on the \textit{paper}
variable which should be sent along with the \textsc{GET} request. If one designates
a non-existing paper, or omits sending the \textit{paper} variable, an error page is
shown.

\paragraph{paperload.c}
This program provides a form where one can fill out author name, paper title, and
select a local file to send to the paperstorage server. By pressing the
\textit{Upload paper} button the paper and other data will be send to the paperstorage
server using a \textsc{POST} request. When all data is present a connection to the
(local) paperstorage server is made, and an \texttt{add} command is executed. If
something goes wrong an error message is shown.

\subsection{Hotel reservations}


\end{document}
