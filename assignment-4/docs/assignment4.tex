\documentclass[a4paper,10pt]{article}


% Title Page
\title{Network Programming\\
\small{Assignment 4: Web Programming}}
\author{%
        \mbox{}\\
        Thomas Veerman\\
        \texttt{tveerman@cs.vu.nl}\\
        1329545\\
        \mbox{}\\
        Martijn Vermaat\\
        \texttt{mvermaat@cs.vu.nl}\\
        1362917
}


\begin{document}
\maketitle

\section{Conference Web site}
\paragraph{Question 1}
We implemented our \textsc{CGI} files in C, the main reason being that
communication with the paper storage server has to be done through \textsc{Sun RPC}
and C is the language with the main \textsc{Sun RPC} support.

Another language frequently used for \textsc{CGI} programming is Perl,
probably because it is very good at manipulating strings and does not have to
be compiled. We briefly considered using Perl with an \texttt{rpcgen}
implementation for Perl or calls directly to C functions (for
example by using the \texttt{Inline::C} package). Our experience with Perl
however is limited and we decided the possible advantage would not be worth the
efforts of getting any of the above methods to work.

\paragraph{Question 2}
The paper storage server has to be accessed through \textsc{Sun RPC}, so this has
to be possible with the language we use. We are not aware of a \texttt{rpcgen}
implementation generating PHP code and talking directly with a \textsc{Sun RPC}
server using PHP socket seems a pointless and time-consuming excercise. The
other option would be to talk to the paper storage server using a different language
and embedding these calls in the PHP code somehow (e.g. by writing a PHP
extension, or executing a compiled C program. This would require some work.

On the other hand, \textsc{CGI} files can be implemented using any language, so in
that case it is easy to pick a language from which we can communicate with the
paper storage server. However, this does not mean \textsc{CGI} is the only way to
deal with this. There are Apache modules for many languages that allow to
execute a piece of code in such a language whenever a request comes in.

\paragraph{Question 3}
We implemented the upload page using a form and a \textsc{POST} request. In general,
\textsc{GET} requests should be used for pages which are idem potent, where 
\textsc{POST} should be used for pages which are not. For example, pages showing
information such as \texttt{papers.cgi} and \texttt{paperview.cgi} are idem potent.
Also, sometimes forms contain private data. Browsers keep \textsc{POST} requests
out of their history file.

Using \textsc{POST} and \textsc{GET} requests are the only methods of sending data
from the client to the server using a WWW browser. When sending form input, you
should in general use \textsc{POST} requests.

\section{Hotel reservations}
\paragraph{Question 4}
It would have been possible to implement the hotel reservation pages in any language
that supports socket programming.

\section{Program documentation}
\subsection{Conference Web site}
For the Conference Website we have written 3 CGI pages:
\begin{center}
% use packages: array
\begin{tabular}{ l | p{9cm} }
\texttt{papers.c} & A page that shows the available papers\\ \hline
\texttt{paperview.c} & A page that makes it possible to download papers\\ \hline
\texttt{paperload.c} & A page that provides a form to upload papers\\
\end{tabular}
\end{center}

\paragraph{papers.c}
This program connects to the (local) paperstorage server and executes a \texttt{list}
command in order to retrieve the list of stored papers. It presents them the same way
as our paperclient of the previous assignment, with the addition of making the titles
of the papers hyperlinks to paperview.cgi so one can download the papers.

\paragraph{paperview.c}
This program connects to the (local) paperstorage server and executes a \texttt{fetch}
command in order to retrieve a specific paper. This file is sent to the browser with
the proper mime type, so the browser can decide how to view the file (or provide a
means to save it somewhere). Which paper it retrieves depends on the \textit{paper}
variable which should be sent along with the \textsc{GET} request. If one designates
a non-existing paper, or omits sending the \textit{paper} variable, an error page is
shown.

Data sent using a \textsc{GET} request is provided to a CGI program in the environment
variable \textsc{QUERY\_STRING}. The contents of this variable is URL encoded, and should
therefore be decoded first.

\paragraph{paperload.c}
This program provides a form where one can fill out author name, paper title, and
select a local file to send to the paperstorage server. By pressing the
\textit{Upload paper} button the paper and other data will be send to the paperstorage
server using a \textsc{POST} request. When all data is present a connection to the
(local) paperstorage server is made, and an \texttt{add} command is executed. If
something goes wrong an error message is shown.

Data sent using a \textsc{POST} request is provided to a CGI program on the standard
input in a format described by RFC 1867. This program checks if a \textsc{POST} request
is done, and parses the standard input accordingly.

\subsection{Hotel reservations}

For the Hotel reservations we have written 6 PHP pages:
\begin{center}
% use packages: array
\begin{tabular}{ l | p{9cm} }
\texttt{site.php} & Configuration\\ \hline
\texttt{conference-shared.php} & Contains functions used in the other pages\\ \hline
\texttt{index.php} & Shows the index of the Conference website\\ \hline
\texttt{hotellist.php} & Lists available rooms and their prices\\ \hline
\texttt{participants.php} & Lists registered participants\\ \hline
\texttt{hotelbook.php} & Provides a form for booking a room\\
\end{tabular}
\end{center}

\paragraph{site.php}
Contains some configuration options, such as the location of the CGI files
and the  server.

\paragraph{conference-shared.php}
Because more than one page needs to connect to the hotel server gateway and
send requests to it, this functionality is abstracted to some functions and
shared by the other pages.

The file also contains some functions that create the HTML layout for all
pages.

\paragraph{index.php}
This page serves as a gateway to all other PHP and CGI pages and therefore
only contains some links.

\paragraph{hotellist.php}
A connection to the hotel server gateway is made and a request for a listing
of available rooms is made. The result is presented to the user with a
convenient link for each type of room to the booking form.

\paragraph{participants.php}
Analoguous to the \texttt{hotellist.php} page, this page requests the hotel
server gateway for a listing of registered participants and presents it to
the user.

\paragraph{hotelbook.php}
The \texttt{hotelbook.php} page handles two types of HTTP requests. The first
is an ordinary \textsc{GET} request which is responded to by showing a booking
form. An optional query string parameter \texttt{type} is checked for a valid
room type and if present, the relevant form field is initialized to this value.
Submitting the form will result in a \textsc{POST} request being sent to the same
page containing the form field values entered by the user.

This \textsc{POST} request is the second type of request which is handled by
the \texttt{hotelbook.php} page. All form field values are validated and if
everything is correct, a booking request is sent to the hotel server gateway. If
there was an error in the submitted form field values, the form is shown again
with the relevant error message. Now the user can try again.


\end{document}
