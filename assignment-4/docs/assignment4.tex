\documentclass[a4paper,10pt]{article}


% Title Page
\title{Network Programming\\
\small{Assignment 4: Web Programming}}
\author{%
        \mbox{}\\
        Thomas Veerman\\
        \texttt{tveerman@cs.vu.nl}\\
        1329545\\
        \mbox{}\\
        Martijn Vermaat\\
        \texttt{mvermaat@cs.vu.nl}\\
        1362917
}


\begin{document}
\maketitle

\section{Conference Web site}
\paragraph{Question 1}
We implemented our \textsc{CGI} files in C, the main reason being that
communication with the paper storage server has to be done through \textsc{Sun RPC}
and C is the language with the main \textsc{Sun RPC} support.

Another language frequently used for \textsc{CGI} programming is Perl,
probably because it is very good at manipulating strings and does not have to
be compiled. We briefly considered using Perl with an \texttt{rpcgen}
implementation for Perl or calls directly to C functions (for
example by using the \texttt{Inline::C} package). Our experience with Perl
however is limited and we decided the possible advantage would not be worth the
efforts of getting any of the above methods to work.

\paragraph{Question 2}
The paper storage server has to be accessed through \textsc{Sun RPC}, so this has
to be possible with the language we use. We are not aware of a \texttt{rpcgen}
implementation generating PHP code and talking directly with a \textsc{Sun RPC}
server using PHP socket seems a pointless and time-consuming excercise. The
other option would be to talk to the paper storage server using a different language
and embedding these calls in the PHP code somehow (e.g. by writing a PHP
extension, or executing a compiled C program. This would require some work.

On the other hand, \textsc{CGI} files can be implemented using any language, so in
that case it is easy to pick a language from which we can communicate with the
paper storage server. However, this does not mean \textsc{CGI} is the only way to
deal with this. There are Apache modules for many languages that allow to
execute a piece of code in such a language whenever a request comes in.

\paragraph{Question 3}
Thomas? :-)

\section{Hotel reservations}
\paragraph{Question 4}
It would have been possible to implement the hotel reservation pages in any language
that supports socket programming.

\section{Program documentation}
\subsection{Conference Web site}

\subsection{Hotel reservations}


\end{document}
