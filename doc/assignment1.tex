\documentclass[a4paper,10pt]{article}


% Title Page
\title{Unix Processing Assignment}
\author{%
        \mbox{}\\
	Thomas Veerman\\
	\texttt{tveerman@cs.vu.nl}\\
	1329545\\
	\mbox{}\\
	Martijn Vermaat\\
	\texttt{mvermaat@cs.vu.nl}\\
	1362917
}


\begin{document}
\maketitle

\section{Micro-shell}
\paragraph{Question 1}
Our shell has to create two child processes after receiving a piped command, one process for the writing command and one process for the reading command. One pipe has to be created for these processes to communicate (only in one direction). The parent shell process waits for both child processes to finish before reading the next command.

\paragraph{Question 2}
This is not possible, because the necessary exec() call will destroy the entire shell process (including any threads).

\paragraph{Question 3}
You cannot use the cd command in our shell. The cd commands in shells are built-in functions of these shells and not executable files.

\section{Synchronization}
\paragraph{Question 4}
Mutual exclusion is needed here, because only one process at a time may be calling the display() function. Mutual exclusion can be implemented with a semaphore by initializing the semaphore to 1 and trying to DOWN() the semaphore before entering the mutual exclusive code and do an UP() afterwards. This way, only one process can be executing the mutual exclusive code at a
time because DOWN() on a semaphore with value 0 will be blocking until UP() is called again.
\begin{tabbing}
\hspace{20pt}\=\kill
 \> DOWN()\\ 
 \> mutex\_code()\\ 
 \> UP() 
\end{tabbing}


\paragraph{Question 5}
The difference with the previous exercise is that the two mutual exclusive code blocks have to be executed alternatingly. In this case, events can be used by a process to wait for authorization and to authorize the other process to continue. Again, this can be implemented using semaphores, but this time we need two of them. The first semaphore implements the event that the first process may continue, while the second semaphore implements the event that the second process may continue. Each process authorizes the other process to continue execution by sending the appropriate event after it has executed the relevant part of his code.
\begin{tabbing}
\hspace{20pt}\=\kill
 \> DOWN(my\_event)\\ 
 \> mutex\_code()\\ 
 \> UP(his\_event) 
\end{tabbing}

\section{pthreads}
\paragraph{Question 6}
We could also have used semaphores in the pthread version, just like we did in the version with multiple processes. However, it would be a strange thing to do: allocate shared memory for a semaphore while the two threads share most of their variables anyway.

\section{Java threads}
\paragraph{Question 7}
Yes. With semaphores one can do mutual exclusion, but also wait for events. In the Java version of the programs we could have simulated syn2.c by using two Semaphore objects, initialized to zero permits, and use \emph{acquire()} and \emph{release()} as DOWN() and UP() respectively. Both threads should be given a reference to these objects so that they both have access to them.

\section{Program documentation}
For the assignments we have written a total of 10 programs:
\begin{center}
% use packages: array
\begin{tabular}{ l | p{9cm} }
mysh1.c & micro-shell that can execute a command (program) \\ \hline
mysh2.c & extension of mysh1.c where a program can have additional parameters \\ \hline
mysh3.c & extension of mysh2.c such that it accepts two piped commands \\ \hline
mysh4.c & extension of mysh3.c that supports the \emph{cd} command \\ \hline
syn1.c & program that uses two processes with synchronization to print \emph{bonjour monde} and \emph{hello world} \\ \hline
syn2.c & program that uses two processes with synchronization to print \emph{abcd} \\ \hline
synthread1.c & same as syn1.c, but uses threads instead of processes \\ \hline
synthread2.c & same as syn2.c, but uses threads instead of processes \\ \hline
syn1.java & same as syn1.c, but implemented in Java \\ \hline
syn2.java & same as syn2.c, but implemented in Java
\end{tabular}
\end{center}
\paragraph{mysh1.c} The program starts by typing ./mysh1 if you are in the same directory as the binary program. A micro-shell will start that will run a program by typing the name of the program (or path to the program) and pressing $<$enter$>$. The shell does not support parameters, because it will treat the complete string as if it is the program name. For example, when typing \emph{ls -la}$<$enter$>$, \emph{ls -la} will be used as program name, which probably does not exist.

The program is designed as follows. The program continuously reads the input the user types in a while loop. When the user has pressed the enter key, the input is stored in an expanding buffer (so that it will always fit). Subsequently an array is built which will ultimately be passed to \emph{execvp}. The array consists of two data elements:
\begin{tabbing}
\hspace{20pt}\=\kill
 \> [0] pointer to char array holding the program name\\ 
 \> [1] 0 - marking the end of the array
\end{tabbing}
The array is checked if the user entered \emph{exit}, which will cause the program to break out of the while loop and terminate (the same happens if the user sent EOF by pressing $<$Ctrl$>$ + $<$D$>$), because \emph{exit} is not a program, but an internal shell command. Finally, the array is passed to a routine \emph{execute\_command} which extracts the program name out of the array, forks off a child process, and calls \emph{execvp} with as parameters the program name and array. Subsequently the parent process returns and does \emph{wait} for the child to exit, and enters a new iteration of the while loop.

We could have designed mysh1.c without an expanding buffer by making a buffer of 100 bytes, which will probably be enough to hold the input. However, this will easily lead to buffer overflows. An alternative is the use of \emph{getline}, a routine part of the GNU extensions. As an alternative to \emph{execvp} we could have used \emph{execlp} because the array seems a bit over the top considering there is no support for parameters. However, in order to make it easy to extend the program as needed in the next couple of assignments, we used \emph{execvp}. Also, we wanted support for the \$PATH variable. For example, typing \emph{ls} will effectively execute \emph{/bin/ls}, even though the user did not enter the /bin/ part.

\paragraph{mysh2.c} The program starts by typing ./mysh2 if you are in the same directory as the binary program. A micro-shell will start that will run a program by typing the name of the program (or path to the program), a few optional parameters for the program, and pressing $<$enter$>$.

The program is designed as follows. In essence it is the same as mysh1.c, but with a few additions. The while loop initially counts how many program parameters are entered, i.e. counting the number of space delimited tokens (multiple consecutive spaces are treated as one). Then instead of building a two data elements array, it takes the number of tokens into account. The array will look like this when the user entered \emph{ls -l -a}:
\begin{tabbing}
\hspace{20pt}\=\kill
 \> [0] pointer to char array holding \emph{ls}\\ 
 \> [1] pointer to char array holding \emph{-l}\\
 \> [2] pointer to char array holding \emph{-a}\\
 \> [3] 0 - marking the end of the array
\end{tabbing}
Then the parameters are copied to the array, and \emph{execute\_command} is called.
A nice feature of our design is that we only needed to extend a few parts of mysh1.c in order to support parameters.

\paragraph{mysh3.c} The program starts by typing ./mysh3 if you are in the same directory as the binary program. A micro-shell will start that will run a program by typing the name of the program (or path to the program), a few optional parameters for the program, and pressing $<$enter$>$. Alternatively, a program can be run by typing the program name with a few optional parameters, a pipe sign $\mid$ , and again a program name with a few optional parameters. For example, it is possible to type a command like \emph{ls -la $\mid$ grep .c}.

The program is designed as follows. It is the same as mysh2.c, but we modified the \emph{execute\_command} routine to accept a couple of parameters describing pipes (i.e. a pipe file descriptor and the usage of the pipe - read or write to it). When the while loop has finished counting tokens and creating and filling the array that is to be passed to \emph{execvp}, a routine is called that checks the array if it contains the pipe sign. If that is the case, it overwrites the pipe sign with \emph{null} -delimiting the array-, creates a (globally accessible) pipe, and returns a pointer to the first data element of the array that belongs to the second program. For example, if the user entered \emph{ls -la $\mid$ grep .c}, a pointer to the data element holding \emph{grep} will be returned. Before checking for pipe signs:
\begin{tabbing}
\hspace{20pt}\=\kill
 \> [0] pointer to char array holding \emph{ls}\\ 
 \> [1] pointer to char array holding \emph{-l}\\
 \> [2] pointer to char array holding \emph{-a}\\
 \> [3] pointer to char array holding \emph{$\mid$}\\ 
 \> [4] pointer to char array holding \emph{grep}\\
 \> [5] pointer to char array holding \emph{.c}\\
 \> [6] 0 - marking the end of the array
\end{tabbing}
After checking for pipe signs:
\begin{tabbing}
\hspace{20pt}\=\kill
 \> [0] pointer to char array holding \emph{ls}\\ 
 \> [1] pointer to char array holding \emph{-l}\\
 \> [2] pointer to char array holding \emph{-a}\\
 \> [3] 0 - marking the end of the array\\ 
 \> [4] pointer to char array holding \emph{grep}\\
 \> [5] pointer to char array holding \emph{.c}\\
 \> [6] 0 - marking the end of the array
\end{tabbing}
If there indeed is a pipe, two calls to \emph{execute\_command} are made. One call holding a pointer to the first program and as pipe action \emph{write to pipe}, and a second call holding a pointer to the second program and as pipe action \emph{read from pipe}. When there is no pipe, the pipe action is \emph{no pipe}, indicating to \emph{execute\_command} that it does not have to do any pipe administration. Otherwise \emph{execute\_command} closes in the case of \emph{write to pipe} after the fork file descriptor 0 of the pipe and \emph{standard out}, and ties \emph{standard out} to file descriptor 1 of the pipe, such that writes to \emph{standard out} are in fact written to the pipe. In the case of \emph{read from pipe}, \emph{execute\_command} closes after the fork file descriptor 1 of the pipe and \emph{standard in}, and ties \emph{standard in} to file descriptor 0 of the pipe, such that reads from \emph{standard in} are in fact read from the pipe. The parent of both child processes closes both pipe file descriptors, because it does not need them (and therefore should be closed!). When all pipe administration is done, a call to \emph{execvp} is made.

It is important to note that the parent has to fork both children, i.e. it does not work if the parent forks a child, and the child forks the next child, because the first child has to do a \emph{wait} on the second child, but the second child is waiting for the first child to execute the command.

\paragraph{mysh4.c} The program starts by typing ./mysh4 if you are in the same directory as the binary program. A micro-shell will start that will run a program by typing the name of the program (or path to the program), a few optional parameters for the program, and pressing $<$enter$>$. Alternatively, a program can be run by typing the program name with a few optional parameters, a pipe sign $\mid$ , and again a program name with a few optional parameters. For example, it is possible to type a command like \emph{ls -la $\mid$ grep .c}. Additionally, the current working directory can be changed by using the \emph{cd} command. Just typing \emph{cd} will change to the home directory of the current user, otherwise an attempt is made to change to the indicated directory (which can fail due file system limitations or non-existing directories).

The program is designed as follows. It is the same as mysh3.c with a slight modification to the main while loop. After the loop checks if the user entered \emph{exit} a check is done if the user entered \emph{cd}. If that is true, it tries to perform the command, and continue to the next iteration of the while loop. The command fails if the file system does not permit to change to a particular directory (due to rights restrictions or non-existent directories) or if it fails to change to the user's home directory (by typing only \emph{cd}, omitting any parameters) because the ENVIRONMENT\_HOMEDIR variable could not be found.

\paragraph{syn1.c} The program starts by typing ./syn1 if you are in the same directory as the binary program. The program will print \emph{Bonjour monde} and \emph{Hello world} interleaved in a random fashion.

The program is designed as follows. The program sets up a semaphore in shared memory, initializes it to 1 for mutual exclusion and then forks. The parent prints ten times \emph{Hello world} and the child prints ten times \emph{Bonjour monde} by using the \emph{display} function. However, before either process calls \emph{display}, they each try to obtain mutual exclusion on the semaphore. One of them obtains it and is able to print its string and then releases the mutual exclusion. This way it is guaranteed that both print the complete string and not mixed up.

\paragraph{syn2.c} The program starts by typing ./syn2 if you are in the same directory as the binary program. The program will print \emph{abcd} ten times.

The program is designed as follows. The program sets up two semaphores in shared memory, initialized to zero for events, and then forks. The parent blocks itself on the event that it is okay to print \emph{ab}, and the child triggers the event that it is okay to print \emph{ab}, and subsequently blocks on the event to print \emph{cd$<$newline$>$}. This construction ensures that it does not matter in which order the processes are executed and always first print \emph{ab}. If the parent runs first it is obvious that it works, but if the child runs first it also works well. The child sets the semaphore to 1 and then blocks (because the parent could not have triggered the ``print cd'' event yet). When the parent finally runs and does a block on the ``print ab'' event, it will not be blocked because the value $>=$ 1 and prints \emph{ab}. The parent triggers the ``print cd'' event, enters the next iteration of the for loop and blocks on the ``print ab'' event again. The child will print \emph{cd$<$newline$>$}, enter the next iteration of the for loop, triggers ``print ab'' event, and blocks on the ``print cd'' event again. When the parent has printed \emph{ab} for the $10^{th}$ time, it triggers ``print cd'' event, exits the for loop and \emph{waits} for the child to finish. It will not block on the semaphores anymore. The child exits the for loop if it printed \emph{cd$<$newline$>$} for the $10^{th}$ time, and terminates.

An alternative is to write one of the events as a mutual exclusion semaphore (which we have tested and works). However, it seemed more natural to implement it as two events where both processes are waiting for events to be triggered.

\paragraph{synthread1.c} The program starts by typing ./synthread1 if you are in the same directory as the binary program. The program will print \emph{Bonjour monde} and \emph{Hello world} interleaved in a random fashion.

The program is designed as follows. The program sets up a semaphore for treads and a thread which executes a \emph{bonjour} routine printing \emph{Bonjour monde} ten times. After starting the thread the process prints \emph{Hello world} ten times. Both threads try to get a lock on the mutex so that they can print their string undisturbedly. When they are done printing they unlock the mutex. This way it is guaranteed that both print the complete string and not mixed up.

\paragraph{synthread2.c} The program starts by typing ./synthread2 if you are in the same directory as the binary program. The program will print \emph{abcd} ten times.

The program is designed as follows. The sets programs up two mutexes and two condition variables. Essentially the program is designed the same way as syn2.c with events, but there is a small difference. There might be the problem of a race condition since both threads try to write to the same condition variable, so writing to a conditional variable and triggering the event is protected by a mutex. 

\paragraph{syn1.java} The programs starts by typing \emph{java syn1} if you are in the same directory as the syn1.class file. The program will print \emph{Bonjour monde} and \emph{Hello world} interleaved in a random fashion.

The program is designed as follows. Two threads are started which have in their \emph{run} method only a loop that ten times prints the string the threads received in via the constructor. The threads have a static \emph{display} method, which ensures that both threads use the exact same method (otherwise they would have their own copy of display, which would never work properly). The \emph{display} method is also synchronized, so only one thread can access the method at the same time.

\paragraph{syn2.java} The program starts by typing \emph{java syn2} if you are in the same directory as the syn2.class file. The program will print \emph{abcd} ten times.

The program is designed as follows. It has the same event structure as syn2.c. Unfortunately, in Java there are no multiple conditional variables available. In order to make blocking and unblocking of threads available, both threads need to be both in the same synchronized method. However, since it is not allowed to modify the \emph{display} method we decided to make a static Display class, with a synchronized wrapper method around the \emph{display} method. The static class makes sure both threads use the same wrapper method, and the synchronized method makes it possible to block and unblock threads. The \emph{wait} and \emph{notify} methods are structured the same way as the event semaphores in syn2.c

\end{document}
