\documentclass[a4paper,10pt]{article}


% Title Page
\title{Network Programming\\
\small{Assignment 3: Client-Server programming}}
\author{%
        \mbox{}\\
        Thomas Veerman\\
        \texttt{tveerman@cs.vu.nl}\\
        1329545\\
        \mbox{}\\
        Martijn Vermaat\\
        \texttt{mvermaat@cs.vu.nl}\\
        1362917
}


\begin{document}
\maketitle

\section{A Paper Storage Server}
\paragraph{Question 1}
It is possible to send any number of paper detail structures by using a list:

\begin{tabbing}
\hspace{20pt}\=\kill
 \>\texttt{typedef document document\_list<>;}
\end{tabbing}

However, by omitting the size of the array Sun RPC actually reads that as $2^{32} - 1$, so in fact it is limited. We have chosen to make a linked list:

\begin{tabbing}
\hspace{20pt}\=\kill
 \> \texttt{type}\=\texttt{def} \texttt{struct document\_node *document\_list;}\\ 
 \> \texttt{struct document\_node }\{ \+ \\
 \> \texttt{document item;}\\ 
 \> \texttt{document\_list next;} \- \\
 \> \} 
\end{tabbing}

\paragraph{Question 2}
Normally you define a variable with a fixed size, but than can prove to be troublesome when you have to deal with variable sized data. That is, you will have to cut data up in several chunks when its size is larger than your fixed size variable. This is difficult to get right with Sun RPC. Our solution to this problem was to define an array where we omitted the size, thus defining it as an array of $2^{32} - 1$ bytes (or just under 4 GiB).

\begin{tabbing}
\hspace{20pt}\=\kill
 \>\texttt{typedef opaque data <>;}
\end{tabbing}

This seems to be large enough for transmitting papers.


\section{A Hotel Reservation Server}
\paragraph{Question 3}
Each method parameter should be passed by value, never by reference (if you do try to send by reference everything related to that particular object will be serialized first which can result in megabytes of data if you are not careful with internal references). This is because the variables have to be local if the server wants to manipulate them; it cannot manipulate variables that still reside at the client.

In our solution we send only integers (Java primitive, always used by value) and Strings (immutable object, but serialization is needed) to the server. This is the only data the server really needs to be able to do its job, there is no need to send larger data structures, so we do not risk sending too much data.

As result our server sends back a Set of Strings or a set of Availability objects which in turn consists of 2 integers and a float (all Java primitives), so also here we do not risk sending too much data.

\section{A Hotel Reservation Gateway}
\paragraph{Question 4}
We have used one thread per child as server structure. Though it is unlikely that we will be sending big quantities of data, it can occur that a client is on a very bad connection with high latency and low throughput, which will slow down the server. Also, as we found out first hand, clients that do not follow the protocol may keep the connection open so that other clients cannot connect. By spawning a thread for each incoming connection, this will not be a problem anymore. Of course the server (not the gateway) must support concurrent connections. To be able to do that we made the booking method synchronized, since that is the only method that might cause problems when multiple clients want to book the last room (of a specific type). The other methods only read data, which will never be a problem.

\paragraph{Question 5}
To design a protocol that is suitable for multiple platforms and languages, you need to make sure to keep the differences between platforms and languages in mind. For example, where one platform considers an int as a 16-bit value, others consider it a 32-bit value. In your protocol you have to standardize these differences.

Our solution is to make a plain-text protocol with ASCII character encoding, based on HTTP. This is easy to debug, and virtually every platform knows how to interpret ASCII character encoding. Almost all data are strings, except for the price of a room, which is a float. We used the standard ASCII encoding for a float with precision two (that is, a number followed by a full stop character followed by a number in the range of 00 to 99). (For a detailed description of our protocol, see the program documentation for the hotel gateway.)

\paragraph{Question 6}
The hotelserver obviously runs at the hotel it is serving for. Also the gateway belongs there, since it only serves as a generic means to connect to the hotelserver. The hotelclient belongs to the client that wants to book a room.

\section{Program documentation}
\subsection{A Paper Storage Server}

For the paper storage server we have written 2 programs:
\begin{center}
% use packages: array
\begin{tabular}{ l | p{9cm} }
paperclient.c & a program that connects to the paper server to store papers and retrieve paper info\\ \hline
paperserver.c & an iterative server that accepts papers and provides information about the papers\\
\end{tabular}
\end{center}

In the specification file paperstorage.x our procedures and data types are defined.

\paragraph{paperclient.c}
The program starts by typing ./paperclient followed by a number of parameters. How to use the program is printed by omitting the parameters. When the program starts it parses the parameters and executes the corresponding routine, otherwise it will print an error message.

The add, detail, fetch, and list routines roughly work as follows. A client structure is created, parameters are set, remote procedure is called, and results are printed on standard output. That means that also the fetched paper will be printed out standard out.

You will have to redirect the output if you want to save the file. We have chosen not to save the file in, for example, the local directory because stored papers might have the same filename.

The paper client uses the following parameters:

\begin{tabbing}
\hspace{20pt}\=\kill
 \> \texttt{add <hostname> <author> <title> <filename.}{\texttt{pdf$\vert$doc}}\texttt{>} \\
 \> to add a paper to the paperstorage.
\end{tabbing}

\begin{tabbing}
\hspace{20pt}\=\kill
 \> \texttt{paperclient detail <hostname> <number>} \\
 \> to retrieve details about a specific paper.
\end{tabbing}

\begin{tabbing}
\hspace{20pt}\=\kill
 \> \texttt{paperclient fetch <hostname> <number>} \\
 \> to retrieve a stored paper.
\end{tabbing}

\begin{tabbing}
\hspace{20pt}\=\kill
 \> \texttt{paperclient list <hostname>} \\
 \> to list the stored papers in paperstorage.
\end{tabbing}

\paragraph{paperserver.c}
The program starts by typing ./paperserver. When the program starts it waits for incoming connections (not actually part of the paperserver object file), and handles the requests.

The server keeps an array of pointers to paper\_t structs. If the array becomes full, it is resized. When a paper is added a piece of memory is allocated and paper details and paper file are copied. Then the number under which the paper is stored is sent back to the client.

When a list request comes in a linked list containing information about each paper is contructed and sent back.

The detail and fetch routines of the client are handled by the same procedure on the server. The only difference is whether they use the \emph{sparse} or \emph{detailed} flag. In both cases the author and paper title is sent back, but in the case of fetch also the paper file is copied into the result.

The server uses TCP connections, because the server routines would have to be idem potent if we were to use UDP. This is no problem with the detail, fetch, and list routines, however, we do not want to have a paper stored multiple times with the add routine because of connection problems. 

\paragraph{paperstorage.x}
As mentioned earlier it is possible to use different techniques to transfer variably sized data. Where it is possible to use an array with maximum size, we used a linked list.

When a procedure fails, we wanted to send back information telling why something went wrong. Instead of incorporating every bit of data in a structure, we used the more efficient unions.

\subsection{A Hotel Reservation Server}
Our implementation is tested on GNU/Linux using JDK 1.5.0\_06-b5 and JDK 1.5.0\_09-b03.

For the hotel reservation server we have written 9 classes:
\begin{center}
% use packages: array
\begin{tabular}{ l | p{7.0cm} }
Availability.java & structure holding price and the number of still available rooms for a specific room type\\ \hline
HotelClient.java & client program that connects to the hotel reservation server to book rooms and retrieve information\\ \hline
Hotel.java & interface for the hotel service running at the hotel reservation server\\ \hline
HotelServer.java & the Hotel Reservation Server\\ \hline
NotAvailableException.java & exception that occurs when a wanted room is not available\\ \hline
NotBookedException.java & exception handling an internal error\\ \hline
Room.java & a specific room that hold guest information when it is booked\\ \hline
RoomType.java & structure that holds data about the specific types of rooms\\ \hline
SimpleHotel.java & implementation of the hotel service\\
\end{tabular}
\end{center}

\paragraph{Availability.java}
This class is used in a Set to send information about which types of rooms still have rooms left, and at which price.

\paragraph{HotelClient.java}
The program starts by typing ./hotelclient (which is a shell script wrapper around the actual Java program) followed by a number of parameters. How to use the program is printed by omitting the parameters.

We have designed HotelClient such that it can run on its own, but also that it can be incorporated in a larger software package that needs the book, details, and list functions. For example, to make a graphical wrapper around it.

The HotelClient uses the following parameters:

\begin{tabbing}
\hspace{20pt}\=\kill
 \> \texttt{hotelclient list <hostname>} \\
 \> to list still available rooms.
\end{tabbing}

\begin{tabbing}
\hspace{20pt}\=\kill
 \> \texttt{hotelclient book <hostname> [type] <guest>} \\
 \> to book a room. If type (of room) is omitted a room will be chosen by the server.
\end{tabbing}

\begin{tabbing}
\hspace{20pt}\=\kill
 \> \texttt{hotelclient guests <hostname>} \\
 \> to retrieve a list of guests that have booked a room.
\end{tabbing}

When the client receives an answer it is printed according to the following format:

\begin{center}
% use packages: array
\begin{tabular}{ l | p{9.3cm} }
\textbf{command} & \textbf{format}\\ \hline
list & "\%3d room(s) of type \%s at \%.2f euros per night\textbackslash n", for example, "5 room(s) of type 1 at 150.00 euros per night\textbackslash n"\\ \hline
book & None or "No rooms of type \emph{type} available" when there are no rooms available of a specific type, or"Unfortunately we're full" when type is omitted and every room is booked \\ \hline
guests & "\emph{guest}\textbackslash n", for example, "Guillaume Pierre\textbackslash nThomas Veerman\textbackslash nMartijn Vermaat\textbackslash n" \\
\end{tabular}
\end{center}

\paragraph{Hotel.java}
This is the interface for the hotel service running at the hotel reservation server. This interface is needed because it is obligatory to extend the java.rmi.Remote interface.
 
\paragraph{HotelServer.java}
The program starts by typing ./hotelserver (which is a shell script wrapper around the actual Java program). Before starting the server the \texttt{rmiregistry} should already be running. However, it is also possible to start your own registry using the \texttt{java.rmi.registry.LocateRegistry} and \texttt{java.rmi.registry.Registry} classes.

The internal workings are pretty simple. The registry is started, a SimpleHotel (implementation of the Hotel interface) is instantiated and bound in the registry. The registry and SimpleHotel take care of the rest. 

\paragraph{NotAvailableException.java}
An exception indicating that either the specified room type was not available anymore, or that the hotel is full in the event the client omits sending a specific room type.

\paragraph{NotBookedException.java}
An internal exception that occurs when we try to retrieve the name of a guest of a booked room that is not booked (thus there is no guest name). The client will never notice that this exception occurred.

\paragraph{Room.java}
Structure representing a room. Holds a guest's name when it is booked.

\paragraph{RoomType.java}
Structure representing a room type. Holds a int representing the type and a float representing price. Of course, the hotel server can attach names to the numerical types. We have chosen to use ints because it is easier to implement.

\paragraph{SimpleHotel.java}
Implementation of the Hotel interface. It generates a couple of room types and then takes care of the book, list, and guests, methods. The SimpleHotel could easily be extended to store the information in a database or file.

The bookRoom method (responsible for bookings) is made synchronized in order to support concurrent connections. This will come in handy for the gateway. The other methods do not need to be synchronized since they only \emph{read} data instead of \emph{write}.

\subsection{A Hotel Reservation Gateway}

Our implementation is tested on GNU/Linux using JDK 1.5.0\_06-b5 and JDK 1.5.0\_09-b03.

For the paper storage server we have written 2 programs:
\begin{center}
% use packages: array
\begin{tabular}{ l | p{8.2cm} }
HotelGateway.java & a program that accepts incoming socket connections and translates them to the hotel reservation server via RMI to book rooms and retrieve information\\ \hline
hotelgwclient.c & a client program that connects to the gateway via sockets to book rooms and retrieve information\\
\end{tabular}
\end{center}

\paragraph{HotelGateway.java}
The program starts by typing ./hotelgw (which is a shell script wrapper around the actual Java program). When the program starts it waits for incoming socket connections. Upon accepting a socket connection, a thread is spawned that handles the request further. In the mean time the server goes back to waiting for new incoming connections.

The thread processes incoming data according to the protocol we specified. When at some point the client fails to follow the protocol an error message is sent back giving some information that indicates what went wrong. When everything is well the thread connects to the Hotel Server using RMI, and invokes the corresponding methods just like the HotelClient.

The Hotel Gateway assumes it is running on the same host as the Hotel Server and therefore connects to the local RMI Registry. The gateway does not need parameters.

\paragraph{hotelgwclient.c}
The program starts by typing ./hotelgwclient followed by a number of parameters. How to use the program is printed by omitting the parameters. When the program starts it parses the parameters and executes the corresponding routine, otherwise it will print an error message. The parameters that can be passed are exactly the same as for HotelClient.java.

Each of the routines (being "list", "book", and "guests") connects to the gateway, sends the proper procedure to the gateway according to the protocol, and reads the response. If something went wrong (e.g. sending an malformed request or internal error on the server side), the error message is printed. Otherwise the response is parsed and printed using the following formats:

\begin{center}
% use packages: array
\begin{tabular}{ l | p{9.3cm} }
\textbf{command} & \textbf{format}\\ \hline
list & "\%3d room(s) of type \%s at \%.2f euros per night\textbackslash n", for example, "5 room(s) of type 1 at 150.00 euros per night\textbackslash n"\\ \hline
book & None. It succeeds or an error message is printed \\ \hline
guests & "\%s\textbackslash n", for example, "Guillaume Pierre\textbackslash nThomas Veerman\textbackslash nMartijn Vermaat\textbackslash n" \\
\end{tabular}
\end{center}

The client does not need to do host to network conversions, since all data is interpreted byte for byte. Therefore we do not need to concern about byte order.

\paragraph{protocol}
As pointed out in our answer to question 5, we have decided to use a plain-text protocol for communication over the sockets between the gateway and clients. The protocol is as follows:

\begin{tabbing}
\hspace{20pt}\=\kill
 \> a req\=uest is a sequence of ASCII characters: \+\\ 
 \> request := procedure newline parameters newline\\
 \> procedure := 'list' $\vert$ 'book' $\vert$ 'guests'\\
 \> parameters := empty $\vert$ ( type newline )? guest newline\\
 \> type := a valid string without newline or space\\ 
 \> guest := a valid string without newline\\
 \> newline := '\textbackslash n' \-\\ 
 \> a response is a sequence of ASCII characters:\+ \\
 \> respone := status space description newline result newline \\
 \> status := '0' $\vert$ '1' $\vert$ '2'\\
 \> space := ' '\\
 \> description := a valid string without newline\\
 \> result := ( guest newline )* $\vert$ ( availability newline )*\\
 \> guest := a valid string without newline\\
 \> availability := type space price space number\\
 \> type := a valid string without newline or space\\
 \> price := float with precision 2 encoded as ASCII string\\
 \> number := an long integer encoded as ASCII string \-\\
 \> where the meaning of stats is: \+\\
 \> 0 - No error, everything is ok\\
 \> 1 - Application error (see result for string description)\\
 \> 2 - Protocol error (usually malformed request)\\
\end{tabbing}




\end{document}
